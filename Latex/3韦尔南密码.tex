\documentclass{article}
\usepackage[UTF8]{ctex}
\usepackage[utf8]{inputenc} % allow utf-8 input
\usepackage[T1]{fontenc}    % use 8-bit T1 fonts
\usepackage{hyperref}       % hyperlinks
\usepackage{url}            % simple URL typesetting
\usepackage{booktabs}       % professional-quality tables
\usepackage{amsfonts}       % blackboard math symbols
\usepackage{nicefrac}       % compact symbols for 1/2, etc.
\usepackage{microtype}      % microtypography
\usepackage{lipsum}
\usepackage{geometry}
\usepackage{graphicx} %插入图片的宏包
\usepackage{float} %设置图片浮动位置的宏包
\usepackage{subfigure} %插入多图时用子图显示的宏包
\usepackage{listings}

\geometry{a4paper,scale=0.8}
\date{}

\usepackage{listings}
\usepackage{color}

\definecolor{dkgreen}{rgb}{0,0.6,0}
\definecolor{gray}{rgb}{0.5,0.5,0.5}
\definecolor{mauve}{rgb}{0.58,0,0.82}

\lstset{frame=tb,
  language=Python,
  aboveskip=3mm,
  belowskip=3mm,
  showstringspaces=false,
  columns=flexible,
  basicstyle={\small\ttfamily},
  numbers=none,
  numberstyle=\tiny\color{gray},
  keywordstyle=\color{blue},
  commentstyle=\color{dkgreen},
  stringstyle=\color{mauve},
  breaklines=true,
  breakatwhitespace=true,
  tabsize=3
}

\title{韦尔南密码}


\author{
Vernam Cipher\\
 刘卓\\
 \texttt{ } \\
}

\begin{document}
\maketitle

\section{余商定理}
\textit{余商定理(Quotient-Remainder Theorem)。 给定一个整数$A$和一个正整数$B$。存在整数$q,r$,使得$
A=B \cdot q+r$ 其中$0 \leq r \leq B$。}

\textit{如果$a \bmod m = b \bmod m = r$ ,则a和b是模全等(Congruent Modulo),记做$a \equiv b \bmod m$。如$8 \equiv -8 \bmod 16$}



\textbf{例1}

$$ \underbrace{521}_{A}=  \underbrace{26}_{B} \times  \underbrace{20}_{(\mbox{商}q)} +\underbrace{1}_{(\mbox{余}r)}
$$

$$521 = 1 \bmod 26$$ 

$\hfill\square$ 

\textbf{例2}

\begin{eqnarray}   
\label{eq}
\underbrace{-521}_{A}&=& -26 \times 20 -1 \nonumber \\ 
&=& 26 \times (-20) +  (-1)  \nonumber \\ 
&=& 26 \times (-20) + 26 + (-1) -26 \nonumber \\ 
&=& \underbrace{26}_{B} \times \underbrace{-21}_{q} + \underbrace{25}_{r}\nonumber \\
\nonumber 
\end{eqnarray}

$$-521 = 25 \bmod 26$$ 

$\hfill\square$ 

\textbf{例3}

\begin{eqnarray}   
\label{eq}
785  &=& 521 + 264 \pmod{26} \nonumber \\ 
&=& 1 +4 \pmod{26} \nonumber \\ 
&=& 5 \pmod{26}
\nonumber 
\end{eqnarray} 
$\hfill\square$ 

\clearpage


\textbf{例4}

\begin{eqnarray}   
\label{eq}
139 \times 787 \pmod{26} &=& (26 \times 5 + 9) \times   (26 \times 30 + 7) \pmod{26} \nonumber \\ 
&=& 9 \times 7 \pmod{26} \nonumber \\ 
&=& 63\pmod{26} \nonumber \\ 
&=& 11 \pmod{26}
\nonumber 
\end{eqnarray}

$\hfill\square$ 

\section{加密过程}
\begin{enumerate}
\item 转化明文,从0开始,按字母表顺序给每个字母分别编号;

\setlength{\tabcolsep}{1mm}{
\begin{tabular}{|c|c|c|c|c|c|c|c|c|c|c|c|c|c|c|c|c|c|c|c|c|c|c|c|c|c|}% 通过添加 | 来表示是否需要绘制竖线
\hline  % 在表格最上方绘制横线
0&1&2&3&4&5&6&7&8&9&10&11&12&13&14&15&16&17&18&19&20&21&22&23&24&25\\
\hline  %在第一行和第二行之间绘制横线
A&B&C&D&E&F&G&H&I&J&K&L&M&N&O&P&Q&R&S&T&U&V&W&X&Y&Z\\
\hline % 在表格最下方绘制横、
\end{tabular}
}

\item 选取两组短密钥U 和 V,然后计算长密钥K:

$$K(i) = U(i) + v(i) \bmod 26$$

其中$1 \leq i \leq n$, $n$为明文长度

\item 计算密文$C$:

$$
C(i):=M(i)+K(i) \bmod 26
$$

然后使用对应的字母替代明文字母;

\end{enumerate}

\textbf{例5}

设密钥${U,V} = {(3,1,2),(7,3,8,4,5)}$,加密明文$NO\ MORE\ AMMO$

解:
$$
\begin{array}{|l||c|c|c|c|c|c|c|c|c|c|c|c|c|c|c|}
\hline \mbox{U} & {\color{red}3}  &{\color{red}1} & {\color{red}2}&3 & 1&2 & 3&1 & 2& 3&1 &2& 3&1 &2 \\
\hline \mbox{V} & {\color{blue}7} & {\color{blue}3} &{\color{blue}8} & {\color{blue}4} & {\color{blue}5} & 7&3 &8 &4 &5 & 7&3 &8 & 4&5 \\
\hline K = U+V \bmod 26 &10 &4 &10 &7 &6 &9 &6 &9 &6 &8 &8 &5 &11 &5 &7 \\
\hline
\end{array}\\
$$

$$
\begin{array}{|c||c|c|c|c|c|c|c|c|c|c|}
\hline \text { \mbox{明文} } & \mathrm{N} & 0 & \mathrm{M} & 0 &\mathrm{R} & \mathrm{E} & \mathrm{A} & \mathrm{M} & \mathrm{M} & \mathrm{O}  \\
\hline M & 13 & 14 & 12 & 14 & 17 & 4 & 0 & 12 & 12 & 14 \\
\hline K & 10 & 4 & 10 & 7 & 6 & 9 & 6 & 9 & 6 & 8 \\
\hline C= M + K \bmod 26 & 23 & 18 & 22 & 21 & 23 &13 &6 & 21&18 & 22\\
\hline \text { \mbox{密文} } & X & S & \mathrm{W} & V & X & N&G &V & S&W \\
\hline
\end{array}
$$

$\hfill\square$ 

\clearpage

\textbf{例6}


\begin{lstlisting}
Plaintext = 'MYNAMEISBOB'#字母需要大写
U = [3,1,2,5,5,7,7,5,7]
V = [5,4,2,3,5,6,4,23,99,12]

def lcm(x, y):
   #  最小公倍数
   if x > y:
       greater = x
   else:
       greater = y
 
   while(True):
       if((greater % x == 0) and (greater % y == 0)):
           lcm = greater
           break
       greater += 1
 
   return lcm

K_len = lcm(len(U),len(V))
K = [0 for i in range(K_len)]

for i in range(K_len):
    K[i] = (U[i%len(U)] +  V[i%len(V)])%26

#print(K)#密钥

Ciphertext = ''
for i in range(len(Plaintext)):
    M_i = ord(Plaintext[i]) -65 #大写字母 -65 ASCII表
    K_i = K[i%len(K)]
    C_i = (M_i+K_i)%26
    Ciphertext += chr(C_i+ 65) 
    
print(Ciphertext)
\end{lstlisting}

输出:UDRIWRTUDDH

$\hfill\square$ 

\clearpage
\textbf{例7}

解密例6的加密字符$UDRIWRTUDDH$

\begin{lstlisting}
Ciphertext = 'UDRIWRTUDDH'   #已知
U = [3,1,2,5,5,7,7,5,7]     #已知
V = [5,4,2,3,5,6,4,23,99,12]    #已知

K_len = lcm(len(U),len(V))
K = [0 for i in range(K_len)]

for i in range(K_len):
    K[i] = (U[i%len(U)] +  V[i%len(V)])%26

print(K)#密钥

Plaintext = ''
for i in range(len(Ciphertext)):
    C_i = ord(Ciphertext[i]) -65 #大写字母 -65 ASCII表
    K_i = K[i%len(K)]
    M_i = (C_i - K_i)%26

    Plaintext += chr(M_i+ 65) 
print(Plaintext)
\end{lstlisting}

输出:MYNAMEISBOB

$\hfill\square$ 









































\end{document}

