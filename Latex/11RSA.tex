\documentclass{article}
\usepackage[UTF8]{ctex}
\usepackage[utf8]{inputenc} % allow utf-8 input
\usepackage[T1]{fontenc}    % use 8-bit T1 fonts
\usepackage{hyperref}       % hyperlinks
\usepackage{url}            % simple URL typesetting
\usepackage{booktabs}       % professional-quality tables
\usepackage{amsfonts}       % blackboard math symbols
\usepackage{nicefrac}       % compact symbols for 1/2, etc.
\usepackage{microtype}      % microtypography
\usepackage{lipsum}
\usepackage{geometry}
\usepackage{amssymb}
\usepackage{graphicx} %插入图片的宏包
\usepackage{float} %设置图片浮动位置的宏包
\usepackage{subfigure} %插入多图时用子图显示的宏包
\usepackage{listings}
\usepackage{soul}
\usepackage{tikz}
\usepackage{tikz-qtree}
\usepackage[colorlinks,linkcolor=blue]{hyperref}
\usepackage{tikz}
\usetikzlibrary{positioning, shapes.geometric}

\geometry{a4paper,scale=0.8}
\date{}

\usepackage{listings}
\usepackage{color}

\definecolor{dkgreen}{rgb}{0,0.6,0}
\definecolor{gray}{rgb}{0.5,0.5,0.5}
\definecolor{mauve}{rgb}{0.58,0,0.82}

\lstset{frame=tb,
  language=Python,
  aboveskip=3mm,
  belowskip=3mm,
  showstringspaces=false,
  columns=flexible,
  basicstyle={\small\ttfamily},
  numbers=none,
  numberstyle=\tiny\color{gray},
  keywordstyle=\color{blue},
  commentstyle=\color{dkgreen},
  stringstyle=\color{mauve},
  breaklines=true,
  breakatwhitespace=true,
  tabsize=3
}

\title{RSA加密算法}


\author{
RSA\\
 刘卓\\
 \texttt{ } \\
}

\begin{document}
\maketitle


\section{现代密码学}

现代密码学的假设:

\begin{enumerate}
\item 对手知道正在使用的系统;
\item 对手可以访问任意数量的相应的明文-密文对;
\item 对手可以访问加密转换$E_k(M) = C$中使用的密钥;
\item 安全性是对手不能构造解密变换$D_k(C) = M$。
\end{enumerate}

\textit{\textbf{定义1:}如果反映射$D_k$的构造在理论上非常复杂,以至于我们现在的计算工具无法实现,那么映射$E_k$就是一个trapdoor函数。}

\textbf{例1}

设两个因子p,q,其中

\begin{eqnarray}   
\label{eq}
p &=& 16347336458092538484431338838650908598417836700330 \nonumber \\ 
&&92312181110852389333100104508151212118167511579 \nonumber \\
q &=& 1900871281664822113126851573935413975471896789968 \nonumber \\ 
&&515493666638539088027103802104498957191261465571 \nonumber \\ 
p \cdot q &=&  3107418240490043721350750035888567930037346022842727545720161948823\nonumber \\ 
&&2064405180815045563468296717232867824379162728380334154710731085019  \nonumber \\ 
&&19548529007337724822783525742386454014691736602477652346609\nonumber \\ 
\nonumber 
\end{eqnarray}

尝试因式分解,会发现很难分解出$p$和$q$。
$\hfill\square$ 

\section{数论}

\textit{\textbf{定义2:} Euler $\varphi$ (n) 函数计算在封区间[1,n]内与n没有公因数的整数的个数。也就是说,}
$$
\begin{aligned}
&\varphi(n)=\#\{m \in \mathbb{Z}: 1 \leq m \leq n, \operatorname{gcd}(m, n)=1\}\\
\end{aligned}
$$


$$
\begin{array}{c|ccccccccccccccc}
n & 1 & 2 & 3 & 4 & 5 & 6 & 7 & 8 & 9 & 10 & 11 & 12 & 13 & 14 & 15 \\
\hline \varphi(n) & 1 & 1 & 2 & 2 & 4 & 2 & 6 & 4 & 6 & 4 & 10 & 4 & 12 & 6 & 8
\end{array}
$$

\textit{\textbf{引理1}:如果m,n为正整数,使得gcd(m,n)= 1,则:}
$$\varphi(mn) = \varphi(m)\varphi(n)$$

\textbf{例2}

$$\varphi(55) = \varphi(5 \times 11) = \varphi(5)\varphi(11) = 4 \times 10$$


$\hfill\square$ 

\textit{\textbf{定义3:} 莫比斯公式:
$$
\begin{aligned}
&\mu(d)=\left\{\begin{array}{ll}
(-1)^{k} \mbox{如果d是k个不同质数的乘积} \\
0 & \mbox{其他} 
\end{array}\right.\\
\end{aligned}
$$
$$
\begin{array}{c|ccccccccccccccc}
d & 1 & 2 & 3 & 4 & 5 & 6 & 7 & 8 & 9 & 10 & 11 & 12 & 13 & 14 & 15 \\
\hline \mu(d) & 1 & -1 & -1 & 0 & -1 & 1 & -1 & 0 & 0 & 1 & -1 & 0 & -1 & 1 & 1
\end{array}
$$
}

$4 = 2 \cdot 2$,有相同的质数所得,所以是0,$12 = 2 \cdot 2 \cdot 3$也有相同质数所得,所以是0。$15 = 3 \cdot 5$由不同质数所得,所以是$(-1)^2 = 1$

\textit{\textbf{引理2}:假设m和n没有公因数。进一步假设m是k个不同素数的乘积n是r个不同素数的乘积。那么mn就是k + r不同质数的乘积,即:
}

$$\mu(mn) = (-1)^{k+r} = (-1)^k(-1)^r = \mu(m)\mu(n)$$

\textit{\textbf{定理1}: 如果n是一个正整数,则可以被质因数分解:}
$$n = p_1^{a_1}p_2^{a_2}\cdots p_r^{a_r}$$

\textit{进一步表示为:}
$$\varphi(n) = n \left(1 - \frac{1}{p_1} \right)\left(1 - \frac{1}{p_2}\right) \cdots \left(1 - \frac{1}{p_r}\right)$$


\textbf{例3}

计算 $\varphi(360)$:


\begin{eqnarray}   
\label{eq}
360&=& 2 \cdot 180  \nonumber \\ 
&=&  2  \cdot 2 \cdot 90 \nonumber \\ 
&=& 2  \cdot 2  \cdot 2  \cdot 45  \nonumber \\ 
&=& 2  \cdot 2  \cdot 2  \cdot 3  \cdot 3 \cdot 5 \nonumber \\ 
&=& 2^3 \cdot 3^2 \cdot 5 \nonumber \\ 
\nonumber 
\end{eqnarray}


\begin{eqnarray}   
\label{eq}
\varphi(360)&=& n \left(1 - \frac{1}{p_1} \right)\left(1 - \frac{1}{p_2}\right) \cdots \left(1 - \frac{1}{p_r}\right)  \nonumber \\ 
&=&  360 \left(1 - \frac{1}{2} \right)\left(1 - \frac{1}{3}\right)\left(1 - \frac{1}{5}\right) \nonumber \\ 
&=& 96 \nonumber \\ 
\nonumber 
\end{eqnarray}

\begin{lstlisting}
# 计算欧拉公式值
def gcd(p,q):
    while q != 0:
        p, q = q, p%q
    return p

def is_coprime(x, y):
    return gcd(x, y) == 1

def phi(x):
    if x == 1:
        return 1
    else:
        n = [y for y in range(1,x) if is_coprime(x,y)]
        return len(n)
print(phi(360))
\end{lstlisting}
$\hfill\square$ 

那么$\varphi(n)$和$\mu(d)$有什么关系呢?

$$\varphi(n) = \sum_{d|n}\mu(d)\frac{n}{d}$$

~\\

\textbf{例4}

让$n = p_1^{a_1}p_2^{a_2}$

则$p_1^{i}p_2^{j}$, $0 \leq i \leq a_1$,$0 \leq j \leq a_2$

\begin{itemize}
\item $ d =  p_1^{0}p_2^{0} = 1\Rightarrow	\mu(1) = (-1)^0 = 1$
\item $ d =  p_1^{1}p_2^{0} = p_1\Rightarrow	\mu(p_1) = (-1)^1 = -1$
\item $ d =  p_1^{0}p_2^{1} = p_2\Rightarrow	\mu(p_2) = (-1)^1 = -1$
\item $ d =  p_1^{0}p_2^{1} = p_1p_2\Rightarrow	\mu(p_1p_2) = (-1)^2 = 1$
\end{itemize}

\begin{eqnarray}   
\label{eq}
\varphi(n)&=&  \mu(1) \cdot \frac{n}{1} + \mu(p_1) \cdot \frac{n}{p_1}+ \mu(p_2) \cdot \frac{n}{p_2}+ \mu(p_1p_2) \cdot \frac{n}{p_1p_2}\nonumber \\ 
&=&  n - \frac{n}{p_1} - \frac{n}{p_2} + \frac{n}{p_1p_2} \nonumber \\ 
&=& n\left(1- \frac{1}{p_1}\right)\left(1- \frac{1}{p_2}\right) \nonumber \\ 
\nonumber 
\end{eqnarray}

$\hfill\square$ 

~\\
模运算的两个基本结果:

\begin{enumerate}
\item \textit{\textbf{定理2(费马 Fermat)}:如果p是一个素数a是一个不能被p整除的整数,那么}
$$a^{p-1} \equiv 1 \quad (mod \quad p)$$
\item \textit{\textbf{定理3(欧拉 Euler)}:如果a和n是相对素数的非零整数,那么}
$$a^{\varphi(n)} \equiv 1 \quad (mod \quad n)$$
\end{enumerate} 

~\\

\textbf{例5}
$$
115 = 5 \cdot 23  \Rightarrow \varphi(115) = \varphi(5)\varphi(23) = 4 \cdot 22 = 88
$$
$$
 {\underbrace{12659}_{{\color{red}a}}}^{ \overbrace{88}^{{\color{red}\varphi(n)}}} \equiv 1 \quad mod  \quad \underbrace{115}_{{\color{red}n}}
$$
$\hfill\square$ 

\section{RSA}

RSA加密算法是一种非对称加密算法,在公开密钥加密和电子商业中被广泛使用。RSA是由罗纳德·李维斯特(Ron Rivest)、阿迪·萨莫尔(Adi Shamir)和伦纳德·阿德曼(Leonard Adleman)在1977年一起提出的。当时他们三人都在麻省理工学院工作。RSA 就是他们三人姓氏开头字母拼在一起组成的。

它是基于这样一个假设:两个素数相乘很容易,但将两个素数的乘积分解成质数分量却很难。到目前为止,世界上还没有任何可靠的攻击RSA算法的方式。只要其密钥的长度足够长,用RSA加密的信息实际上是不能被破解的。

\subsection{RSA使用步骤}

\begin{enumerate}
\item (密钥创建,私钥) ${\color{red}Bob}$ 选择两个只有自己知道的质数 p 和 q, 并且$p \ne q$, 然后计算$N = p \cdot q$; 
\item ${\color{red}Bob}$ 选择一个正整数素数的加密指数(encryption exponent)e,计算$\varphi(N) = (p-1)(q-1)$;
\item ${\color{red}Bob}$公开 N 和 e;
\item (加密) ${\color{blue}Alice}$将明文m, 使用${\color{red}Bob}$公开的密钥加密。密文计算方法为$c \equiv m^e \quad (mod \quad N)$;
\item ${\color{blue}Alice}$发送密文c给${\color{red}Bob}$;
\item (解密)${\color{red}Bob}$解开密文c,一共两步:
\begin{enumerate}
\item $d \equiv e^{-1}(\bmod \quad \varphi(N)) \equiv e^{-1}(\bmod \quad  (p-1)(q-1))$
\item $m \equiv c^d (\bmod \quad  N)$
\end{enumerate}
\end{enumerate}

\subsection{RSA安全性}

假设${\color{blue}Eve}$ 已经知道公钥${\color{red}N,e}$ 和密文 ${\color{red}c}$,想要得到d。那么${\color{blue}Eve}$需要尝试解方程
$$d \cdot e \equiv 1 \quad (\bmod \quad \varphi(N))$$

但是仅从N得到$\varphi(N)$等价于分解N,这是非常困难的,因为:


$$\varphi(N)  = (p-1)(q-1)=pq - p -q + = N -(p+q)+1$$

为了得到到$\varphi(N)$,但有两件事困扰着你:
\begin{enumerate}
\item 将$N$分成两个质数$p,q$从目前计算机算力角度来说是非常困难的;
\item $p,q$的和从未公开
\end{enumerate}

虽然难以破解,但假如${\displaystyle N}$的长度小于或等于256位,那么用一台个人电脑在几个小时内就可以暴力因数分解它的因子了。但当N很大时,破解难度和时间就会飞增。现在加密系统中,一般采取2048位长度的${\displaystyle N}$来确保安全性

\subsection{RSA加密方法}

为了发送消息,即一段字符串,我们需要将将消息转换为数字,反之亦然。

基本思想是使用字母数量等于N以26为基数的对数。为了得到的任何k位数字(以26为底)都必须小于N,所以我们必须有
$$N \ge 26^k - 1 \Rightarrow k = \log_26(N)$$

k为整数,向下取整。这里的k就是一次性能发送多少位字符的数量。比如说$N =131 \cdot 1873 = 245363 \Rightarrow k = \log_26(245363) = 3$,也就是说使用质数$p=131,q=1873$时,一次可以加密3个字符的信息发送给对方。

~\\

\textbf{例6}

令明文$m = \textbf{THE}$

那么 $m = {\color{red}\textbf{T}} \cdot 26^0 + {\color{blue}\textbf{H}} \cdot 26^1 +  {\color{green}\textbf{E}} \cdot 26^2 = {\color{red}19} + {\color{blue}7} \cdot 26 + {\color{green}4} \cdot 26^2 = 2905$

又已知$e = 323,N = 245363$,所以$2905^323 \equiv 13388 \quad (\bmod \quad 245363)$

那么如果是以拉丁字母表26个字母为基底的话,加密过程如下:

\begin{eqnarray}   
\label{eq}
13388&=& 24 + 514 \cdot {\color{red}26}   \nonumber \\ 
&=&  24 + (20 + 19 \cdot {\color{red}26} ) \cdot {\color{red}26}  \nonumber \\ 
&=& 24 \cdot 1 + 20 \cdot {\color{red}26} + 19 \cdot {\color{red}26}^2  \nonumber \\ 
&=& \textbf{Y}\cdot {\color{red}26}^0 + \textbf{U}\cdot {\color{red}26}^1 + \textbf{T}\cdot {\color{red}26}^2 \nonumber \\ 
\nonumber 
\end{eqnarray}

所以明文\textbf{THE}通过RSA可以加密为\textbf{YUT}

$\hfill\square$ 

\subsection{幂取模运算Power in modulo arithmetic}

\textbf{定理4:} 如果 a,n,x,y是非负整数并且满足$gcd(a,n) = 1$,则:
$$x \equiv y \quad (\bmod \quad \varphi(n) \Rightarrow a^x \equiv a^y \quad (\bmod \quad n))$$

证明:


\begin{eqnarray}   
\label{eq}
x &\equiv& y \quad (\bmod \quad \varphi(n)  \nonumber \\ 
&\equiv& y + k \cdot  \varphi(n) \nonumber \\ 
&\Rightarrow& a^x = a^{y + k \cdot  \varphi(n)} \nonumber \\ 
&\Rightarrow& a^x = a^{y} + a^{k \cdot  \varphi(n)} \nonumber \\ 
&\Rightarrow& a^x = a^{y} + a^{\varphi(n)}^k \nonumber \\ 
&\Rightarrow& a^x = a^{y} \quad (\bmod \quad n) \nonumber \\ 
\nonumber 
\end{eqnarray}

\textbf{例7}

$$7^{42} \quad (\bmod \quad 11)$$

$$42 \equiv 2  (\bmod \quad \varphi(11)) = 2  (\bmod \quad 10)$$

$$7^{42} \equiv 7^2  (\bmod \quad 11) = 49 (\bmod \quad 11) = 5$$
$\hfill\square$ 

\subsection{计算大数的余}

~\\

\textbf{例8}

如果计算$1915793^{2641} \quad (\bmod \quad 5678923)$这种大数的mod呢?我们不想处理大于$n^2$的数字。应用定理4.

~\\

第一步:

~\\

a = 1915793,k = 5678923,n = 5678923

将k转化为二进制数,

$$k = 5678923 = 1 + 2^4+2^6+2^9+2^{11} = 1+16+64+512+2048$$

\begin{lstlisting}
#二进制数
number = 2641
number_bin = bin(number)[2:]

cov_number_bin = bin(number)[2:][::-1]

sum_number = ''

for i in range(len(cov_number_bin)):
    if cov_number_bin[i] == '1':
        # print(i)
        if sum_number == '':
            sum_number += str(2**(i))
        else:
            sum_number += ' + ' + str(2**(i))
            

print(sum_number)
\end{lstlisting}

~\\

第二步:

~\\

用重复的平方来得到上面的幂的余:

$a^2 \equiv 3278564 \quad (\bmod \quad n)$,得到3278564,就可以继续使用作为$a^4$的基本单位,不用计算$a^4$,只需要计算$a^4 \quad (\bmod \quad n)= (a^2)^2 \quad (\bmod \quad n)= 3278564^2 \quad (\bmod \quad n) \equiv 1631541 \quad (\bmod \quad n)  $。得到1631541后又可以继续应用在$a^8$上,使得$a^8  \quad (\bmod \quad n) = (a^4)^2  \quad (\bmod \quad n) = 1631541^2  \quad (\bmod \quad n) \equiv 4704430  \quad (\bmod \quad n) $。以减少运算量,以此列推,得到所有a的2的幂次项的余。

$$
\begin{array}{lll}
a^{2} \equiv 3278564 \quad (\bmod \quad n) & a^{4} \equiv 1631541 \quad (\bmod \quad n) & a^{8} \equiv 4704430 \quad  \quad (\bmod \quad n) \\
a^{16} \equiv 1424066 \quad (\bmod  \quad n) & a^{32} \equiv 3532287 \quad (\bmod \quad n) & a^{64} \equiv 3305529 \quad (\bmod \quad n) \\
a^{128} \equiv 1529537 \quad (\bmod \quad n) & a^{256} \equiv 5673135 \quad (\bmod \quad n) & a^{512} \equiv 5106329 \quad (\bmod \quad n) \\
a^{1024} \equiv 2627277 \quad (\bmod \quad n) & a^{2048} \equiv 1180227 \quad (\bmod \quad n) & \cdots
\end{array}
$$

~\\

第三步:

~\\

\begin{eqnarray}   
\label{eq}
a^{k} &=& a \cdot a^{16} \cdot a^{64} \cdot a^{512} \cdot a^{2048}  \pmod{n} \nonumber \\ 
 &=& 1915793 \cdot 1424066 \cdot 3305529 \cdot 5106329 \cdot 1180227 \quad (\bmod \quad n) \nonumber \\ 
 &=&  1162684 \quad (\bmod \quad n) \nonumber \\ 
\nonumber 
\end{eqnarray}

完成Python计算过程如下,速度比{\color{red}$a**n\%p$}快,因为无需重复计算。下面函数等同于内置函数 {\color{red}pow(a,n,p)}

\begin{lstlisting}
def fastExpMod(a, n, p):
    result = 1
    while n != 0:
        if (n&1) == 1:
            # ei = 1, then mul
            result = (result * a) % p
        n >>= 1
        # a, a^2, a^4, a^8, ... , a^(2^n)
        a = (a*a) % p
    return result
\end{lstlisting}



$\hfill\square$ 

\subsection{完整的一次RSA加密和解密过程}

\textbf{例9}

选择$p=53,q = 89$,已知加密因子和解密因子$e = 119,d = 424$

首先检查是否满足$e \cdot d \equiv 1 \quad (\bmod \quad \varphi(4717))= 1 \quad (\bmod \quad (53-1)(89-1))  = 1 \quad (\bmod \quad 4576) $

~\\

1. 加密明文{\color{red}75}

第一步:
$$N = p \times q = 53 \cdot 89 =4717$$

第二步:
$$c = m^e \quad (\bmod \quad N) = {\color{red}75}^{119} \quad (\bmod \quad 4717) = 3500$$

第三步:
$$119 = 2^0+2^1+2^2+2^4+2^5+2^6 = 1+2+4+16+32+64$$

第四步:
$$
\begin{array}{|c|c|c|c|c|c|c|c|}
\hline
k & 1 & 2 & 4 & 8 & 16 & 32 & 64 \\
\hline
75^k \quad (\bmod \quad N) & 75 &908&3706&3249&4072&929&4547\\
\hline
\end{array}
$$

$$75^{119} = 75\cdot908\cdot3706\cdot3249\cdot4072\cdot929\cdot4547 \quad (\bmod \quad 4717)  = 3530$$

~\\

2. 解密密文{\color{blue}2}

$$m = c^d \quad (\bmod \quad N) = {\color{blue}2}^{423} \quad (\bmod \quad 4717) = 3414$$

$\hfill\square$ 

\section{应用RSA在ATM机}

我们卡上的磁片可以通过自动柜员机(ATM)与银行联系,我们通过输入的4到8位的密码来操作我们的账户,那么银行是如何保护我们的账户安全的呢?
\begin{enumerate}
\item 银行选择两个大质数${\color{red}p}$和${\color{red}q}$,并计算${\color{green}N} = p \cdot q$;
\item 银行对每张卡进行编程,对每张卡选择不同的加密因子${\color{blue}e}$加密RSA。即每张卡有不一样的加密因子;
\item 顾客办理银行卡时,选择自己喜欢的密码,这个密码通常是4到8位(银行决定位数)的整数,可以不是素数;
\item 银行在客户的文件中存储密码并生成相应的解密因子${\color{red}d}$;
\item 当客户在自动柜员机(ATM)上插入卡并输入密码时,银行就会检索客户的文件,获取一个很大的随机数,这个随机数$X \in [1,{\color{green}N}-1]$,然后计算$X^{\color{red}d} \pmod{{\color{green}N}}$并发送到ATM进行验证;
\item ATM计算$X = (X^{\color{red}d})^{{\color{blue}e}} \pmod{{\color{green}N}}$,并返回$(X+1)^{{\color{blue}e}} \pmod{{\color{green}N}}$给银行;
\item 银行计算$((X+1)^{{\color{blue}e}})^{{\color{red}d}} \pmod{{\color{green}N}}$和$X+1$是否相等,如果相等,则输入密码正确;否则密码错误;
\end{enumerate}

因为攻击者很难从$X^d$ 和$(X+1)^e$找到加密因子$e$和解密因子$d$,因此可以保证账户安全。

\end{document}

