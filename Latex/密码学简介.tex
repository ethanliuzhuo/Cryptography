\documentclass{article}
\usepackage[UTF8]{ctex}
\usepackage[utf8]{inputenc} % allow utf-8 input
\usepackage[T1]{fontenc}    % use 8-bit T1 fonts
\usepackage{hyperref}       % hyperlinks
\usepackage{url}            % simple URL typesetting
\usepackage{booktabs}       % professional-quality tables
\usepackage{amsfonts}       % blackboard math symbols
\usepackage{nicefrac}       % compact symbols for 1/2, etc.
\usepackage{microtype}      % microtypography
\usepackage{lipsum}
\usepackage{geometry}
\geometry{a4paper,scale=0.8}
\date{}


\title{密码学简介}


\author{
 刘卓\\
 \texttt{ } \\
}

\begin{document}
\maketitle

\section{介绍}
密码学(Cryptography)是研究以加密的形式发送信息的方法,只有掌握此加密技术的特定人群才能破解加密获得有效信息。

一些关键词:
\begin{itemize}
\item 明文(Plaintext):加密前的信息。
\item 密文(Ciphertext):加密后的信息。
\item 加密(Encryption): 通过特定的加密技术将明文转化为密文的行为。
\item 解密(Decryption):由掌握加密技术的特定人群将密文转化为明文的行为。
\item 发件人(Sender):$Alice$.
\item 收件人(Receiver): $Bob$.
\item 攻击者\拦截者(Attacker):$Eva$.
\item n-gram:由n个字母组成的字符串。
\item 密钥(Key): 密钥通常是一系列数字或符号,用于确定明文转换为密文的算法。只使用一次的键称为一次性键盘/键。
\item 密钥空间(Key Space): 用$K$表示,密钥的所有可能。
\item 明文(Plaintext ):用$M$表示,明文消息的集合。
\item 密文空间(Ciphertext space):用$C$表示,所有在特定的加密事务中可能出现的明文消息。
\end{itemize}

\clearpage

\section{加密主要方法}
\begin{itemize}
\item \textbf{替换}(Substitution):n个字母的明文替换成n个字母密文。
\item \textbf{换位}(Transposition):将原始信息的字符按照某些特定的模式重新排列。
\end{itemize}

现代的加密方式是以上两种的混合。

\section{加密解密过程}

明文,由\textit{Alice }通过密钥加密,变成密文。通过选定途径,发送给\textit{Bob},\textit{Bob}通过密钥进行解密,得到明文。

\section{密码学假设}
\begin{itemize}
\item 发送方和接收方之间没有安全的通信通道。
\item 信息的安全性是因为攻击者不知道加密中使用了何种特定密钥,以及加密方式。如果攻击者一旦知道密钥及加密方法,密文则被破译。
\item 攻击者有三种攻击方式
\begin{enumerate}
\item 纯密文攻击:攻击者只需要掌握密文来恢复明文。
\item 明文攻击:攻击者通过掌握一些关于原始明文的信息来恢复明文。
\item 选择性明文攻击:攻击者通过获得与其选择的明文对应的密文来部分破解密文。
\end{enumerate}
\end{itemize}

\end{document}